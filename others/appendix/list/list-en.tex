\ifx\wholebook\relax \else
% ------------------------ 

\documentclass{article}
%------------------- Other types of document example ------------------------
%
%\documentclass[twocolumn]{IEEEtran-new}
%\documentclass[12pt,twoside,draft]{IEEEtran}
%\documentstyle[9pt,twocolumn,technote,twoside]{IEEEtran}
%
%-----------------------------------------------------------------------------
%%
% loading packages
%
\newif\ifpdf
\ifx\pdfoutput\undefined % We're not running pdftex
  \pdffalse
\else
  \pdftrue
\fi
%
%
\ifpdf
  \RequirePackage[pdftex,%
            CJKbookmarks,%
       bookmarksnumbered,%
              colorlinks,%
          linkcolor=blue,%
              hyperindex,%
        plainpages=false,%
       pdfstartview=FitH]{hyperref}
\else
  \RequirePackage[dvipdfm,%
             CJKbookmarks,%
        bookmarksnumbered,%
               colorlinks,%
           linkcolor=blue,%
               hyperindex,%
         plainpages=false,%
        pdfstartview=FitH]{hyperref}
  \AtBeginDvi{\special{pdf:tounicode GBK-EUC-UCS2}} % GBK -> Unicode
\fi
\usepackage{hyperref}

% other packages
%-----------------------------------------------------------------------------
\usepackage{graphicx, color}
\usepackage{CJK}
%
% for programming 
%
\usepackage{verbatim}
\usepackage{listings}


\lstdefinelanguage{Smalltalk}{
  morekeywords={self,super,true,false,nil,thisContext}, % This is overkill
  morestring=[d]',
  morecomment=[s]{"}{"},
  alsoletter={\#:},
  escapechar={!},
  literate=
    {BANG}{!}1
    {UNDERSCORE}{\_}1
    {\\st}{Smalltalk}9 % convenience -- in case \st occurs in code
    % {'}{{\textquotesingle}}1 % replaced by upquote=true in \lstset
    {_}{{$\leftarrow$}}1
    {>>>}{{\sep}}1
    {^}{{$\uparrow$}}1
    {~}{{$\sim$}}1
    {-}{{\sf -\hspace{-0.13em}-}}1  % the goal is to make - the same width as +
    %{+}{\raisebox{0.08ex}{+}}1		% and to raise + off the baseline to match -
    {-->}{{\quad$\longrightarrow$\quad}}3
	, % Don't forget the comma at the end!
  tabsize=2
}[keywords,comments,strings]

\lstloadlanguages{C++, Lisp, Haskell, Python, Smalltalk}

% ======================================================================

\def\BibTeX{{\rm B\kern-.05em{\sc i\kern-.025em b}\kern-.08em
    T\kern-.1667em\lower.7ex\hbox{E}\kern-.125emX}}

\newtheorem{theorem}{Theorem}

%
% mathematics
%
\newcommand{\be}{\begin{equation}}
\newcommand{\ee}{\end{equation}}
\newcommand{\bmat}[1]{\left( \begin{array}{#1} }
\newcommand{\emat}{\end{array} \right) }
\newcommand{\VEC}[1]{\mbox{\boldmath $#1$}}

% numbered equation array
\newcommand{\bea}{\begin{eqnarray}}
\newcommand{\eea}{\end{eqnarray}}

% equation array not numbered
\newcommand{\bean}{\begin{eqnarray*}}
\newcommand{\eean}{\end{eqnarray*}}

\RequirePackage{CJK,CJKnumb,CJKulem,CJKpunct}
% we use CJK as default environment
\AtBeginDocument{\begin{CJK*}{GBK}{song}\CJKtilde\CJKindent\CJKcaption{GB}}
\AtEndDocument{\clearpage\end{CJK*}}

%
% loading packages
%

\RequirePackage{ifpdf}

%
%
\ifpdf
  \RequirePackage[pdftex,%
       bookmarksnumbered,%
              colorlinks,%
          linkcolor=blue,%
              hyperindex,%
        plainpages=false,%
       pdfstartview=FitH]{hyperref}
\else
  \RequirePackage[dvipdfm,%
        bookmarksnumbered,%
               colorlinks,%
           linkcolor=blue,%
               hyperindex,%
         plainpages=false,%
        pdfstartview=FitH]{hyperref}
\fi
\usepackage{hyperref}

% other packages
%--------------------------------------------------------------------------
\usepackage{graphicx, color}
\usepackage{subfig}

\usepackage{amsmath, amsthm, amssymb} % for math
\usepackage{exercise} % for exercise

%
% for programming 
%
\usepackage{verbatim}
\usepackage{listings}
%\usepackage{algorithmic} %old version; we can use algorithmicx instead
\usepackage{algorithm} 
\usepackage[noend]{algpseudocode} %for pseudo code, include algorithmicsx automatically
\usepackage{makeidx} % for index support


\lstdefinelanguage{Smalltalk}{
  morekeywords={self,super,true,false,nil,thisContext}, % This is overkill
  morestring=[d]',
  morecomment=[s]{"}{"},
  alsoletter={\#:},
  escapechar={!},
  literate=
    {BANG}{!}1
    {UNDERSCORE}{\_}1
    {\\st}{Smalltalk}9 % convenience -- in case \st occurs in code
    % {'}{{\textquotesingle}}1 % replaced by upquote=true in \lstset
    {_}{{$\leftarrow$}}1
    {>>>}{{\sep}}1
    {^}{{$\uparrow$}}1
    {~}{{$\sim$}}1
    {-}{{\sf -\hspace{-0.13em}-}}1  % the goal is to make - the same width as +
    %{+}{\raisebox{0.08ex}{+}}1		% and to raise + off the baseline to match -
    {-->}{{\quad$\longrightarrow$\quad}}3
	, % Don't forget the comma at the end!
  tabsize=2
}[keywords,comments,strings]

\lstloadlanguages{C++, Lisp, Haskell, Python, Smalltalk}

% ======================================================================

\def\BibTeX{{\rm B\kern-.05em{\sc i\kern-.025em b}\kern-.08em
    T\kern-.1667em\lower.7ex\hbox{E}\kern-.125emX}}

%
% mathematics
%
\newcommand{\be}{\begin{equation}}
\newcommand{\ee}{\end{equation}}
\newcommand{\bmat}[1]{\left( \begin{array}{#1} }
\newcommand{\emat}{\end{array} \right) }
\newcommand{\VEC}[1]{\mbox{\boldmath $#1$}}

% numbered equation array
\newcommand{\bea}{\begin{eqnarray}}
\newcommand{\eea}{\end{eqnarray}}

% equation array not numbered
\newcommand{\bean}{\begin{eqnarray*}}
\newcommand{\eean}{\end{eqnarray*}}

\newtheorem{theorem}{Theorem}[section]
\newtheorem{lemma}[theorem]{Lemma}
\newtheorem{proposition}[theorem]{Proposition}
\newtheorem{corollary}[theorem]{Corollary}


\setcounter{page}{1}

\begin{document}

\fi
%--------------------------

% ================================================================
%                 COVER PAGE
% ================================================================

\title{Lists}

\author{Liu~Xinyu
\thanks{{\bfseries Liu Xinyu } \newline
  Email: liuxinyu95@gmail.com \newline}
  }

\markboth{Sequences}{AlgoXY}

\maketitle

\ifx\wholebook\relax
\chapter{Lists}
\numberwithin{Exercise}{chapter}
\fi

% ================================================================
%                 Introduction
% ================================================================
\section{Introduction}
\label{introduction}
This book intensely uses recursive list manipulations in purely functional settings.
List can be treated as a counterpart to arrays in imperative settings, which is
bricks to many algorithms and data structures.

For the readers who are not familiar with functional list manipulation, this appendix
provides a quick reference. All operations listed in this appendix are not only
described in equations, but also implemented in both functional programming languages
as well as imperative languages examples. We also provide a special type of
implementation in C++ template meta programming similar to \cite{moderncxx}
for interesting.

Besides the elementary list operations, this appendix also contains explaination of 
some high order function concepts such as mapping, folding etc.


% ================================================================
%                 Binary random access list
% ================================================================
\section{List Definition}
\index{List!Definition}

Like arrays in imperative settings, lists play a critical role in functional setting\footnote{Some 
reader may argue that `lambda calculus plays the most critical role'.
Lambda calculus is somewhat as assembly languages to the computation world, which 
is worthy studying from the essense of computation model to the practical programs.
However, we don't dive into the topic in this book. Users can refer to \cite{mittype}
for detail.}. Lists are built-in supported in some programming languages like Lisp
families and ML families so it needn't explicitly define list in those environment.

List, or more precisely, singly linked-list is a data strucutre that can be described
below.

\begin{itemize}
\item A {\em list} is either empty;
\item Or contains an element and a {\em list}.
\end{itemize}

Note that this definition is recursive. Figure \ref{fig:list-example} illustrates
a list with $N$ nodes. Each node contains two part, a key element and a sub list. The
sub list contained in the last node is empty, which is denoted as 'NIL'.

\begin{figure}[htbp]
        \centering
        \includegraphics[scale=0.8]{img/list-example.ps}
        \caption{A list contains $N$ nodes} \label{fig:list-example}
\end{figure}

This data structure can be explictly defined in programming languages support record
(or compound type) concept. The following ISO C++ code defines list.

\lstset{language=C++}
\begin{lstlisting}
template<typename T>
struct List {
  T key;
  List* next;
};
\end{lstlisting}

\subsection{Empty list}
It's worth to metion about 'empty' list a bit more in detail. In environment supporting the
nil concept, for example, C or java like programming languages, empty list can have two
different represention. One is the trivial `NIL' (or null, 0 varies from languages);
the other is an non-NIL empty list as $\{ \}$, the latter is typically allocated with
memory but filled with nothing. In Lisp dialects, the empty is commonly written as \verb|'()|.
In ML families, it's written as \verb|[]|. We use $\Phi$ to denote empty list in equations
and use 'NIL' in pseudo code to describe algorithms in this book.

\subsection{Access the element and the sub list}
Given a list $L$, two functions can be defined to access the element stored in it and
the sub list respectively. They are typically denoted as $first(L)$, and $rest(L)$ or
$head(L)$ and $tail(L)$ for the same meaning.
These two functions are named as \verb|car| and \verb|cdr| in Lisp for historic reason
about the design of machine registers \cite{SICP}. In languages support Pattern matching (e.g. ML families)
These two functions are commonly realized by matching the \verb|cons| which we'll introduced
later. for example the following Haskell program:

\lstset{language=Haskell}
\begin{lstlisting}
head (x:xs) = x
tail (x:xs) = xs
\end{lstlisting}

If the list is defined in record syntax like what we did above, these two functions can 
be realized by accessing the record fields \footnote{They can be also named as 'key' and 'next'
or be defined as class methods.}.

\lstset{language=C++}
\begin{lstlisting}
template<typename T>
T first(List<T> *xs) { return xs->key; }

template<typename T>
List<T>* rest(List<T>* xs) { return xs->next; }
\end{lstlisting}

In this book, $L'$ is used to denote the $rest(L)$ sometimes, also we uses $l_1$ to represent
$first(L)$ in the context that the list is literately given in form $L = \{ l_1, l_2, ..., l_N\}$.

More interesting, as far as in an environment support recursion, we can define List. The following
example define a list of integers in C++ compile time.

\lstset{language=C++}
\begin{lstlisting}
struct Empty;

template<int x, typename T> struct List {
  static const int first = x;
  typedef T rest;
};
\end{lstlisting}

These line constructs a list of $\{1, 2, 3, 4, 5\}$ in compile time.

\begin{lstlisting}
typedef List<1, List<2, List<3, List<4 List<5, Empty> > > > > A;
\end{lstlisting}

\section{Basic list manipulation}

\subsection{Construction}

The last C++ template meta programming example actually shows literate constraction of a list.
A list can be constructed from an element with a sub list, where the sub list can be empty.
We denote function $cons(x, L)$ as the constructor. This name is used in most Lisp dialects.
In ML families, there are `cons' operator defined as \verb|::|, (in Haskell it's \verb|:|).

We can define \verb|cons| to create a record as we defined above in ISO C++, for example\footnote{
It's often defined as a constructor method for the class template, However, we define it as a standalone
function for illustration purpose.}.

\lstset{language=C++}
\begin{lstlisting}
template<typename T>
List<T>* cons(T x, List<T>* xs) {
  List<T>* lst = new List<T>;
  lst->key = x;
  lst->next = xs;
  return lst;
}
\end{lstlisting}

\subsection{Empty testing and length calculating}

Testing if a list is empty is trivial. If the environment contains nil concept, the testing should
also handle nil case. Both Lisp dialects and ML families provide \verb|null| testing functions.
Empty testing can also be realized by pattern-matching with empty list if possible. The following
Haskell program shows such example.

\lstset{language=Haskell}
\begin{lstlisting}
null [] = True
null _ = False
\end{lstlisting}

In this book we will either use $empty(L)$ or $L = \Phi$ where empty testing happens.

With empty testing defined, it's possible to calculate length for a list. 
In imperative settings, \textproc{Length} is ofthen implemented like the following.

\begin{algorithmic}
\Function{Length}{L}
  \State $n \gets 0$
  \While{$L \neq NIL$}
    \State $n \gets n + 1$
    \State $L \gets $ \Call{Next}{$L$}
  \EndWhile
\EndFunction
\end{algorithmic}

This ISO C++ code translates the algorithm to real program.

\lstset{language=C++}
\begin{lstlisting}
template<typename T>
int length(List<T>* xs) {
  int n = 0;
  for (; xs; ++n, xs = xs->next);
  return n
}
\end{lstlisting}

However, in purely funcitonal setting, we can't mutate a counter variable. 
the idea is that, if the list is empty, then its size is zero; otherwise, we can recursively 
calculate the length of the sub list, then add it by one to get the length of this list.

\be
length(L) = \left \{
  \begin{array}
  {r@{\quad:\quad}l}
  0 & L = \Phi \\
  1 + length(L') & otherwise
  \end{array}
\right.
\ee 

Here $L' = rest(L)$ as mentioned above, it's $\{l_2, l_3, ..., l_N\}$ for list contains $N$ elements.
Note that both $L$ and $L'$ can be empty $\Phi$. In this equation, we also use $=$ to test if list
$L$ is empty. In order to know the length of a list, we need traverse all the elements from the head
to the end, so that this algorithm is proprotion to the number of elements stored in the list.
It's a linear $O(N)$ algorithm.

Below are two programs in Haskell and in Scheme/Lisp realize this recursive algorithm.

\lstset{language=Haskell}
\begin{lstlisting}
length [] = 0
length (x:xs) = 1 + length xs
\end{lstlisting}

\lstset{language=Lisp}
\begin{lstlisting}
(define (length lst)
  (if (null? lst) 0 (length (cdr lst))))
\end{lstlisting}

The length calculation can also be implemented in C++ template meta programming in purely recursive
manner.

\lstset{language=C++}
\begin{lstlisting}
template<typename L> struct Length {
  static const int value = 1 + Length<typename L::rest>::value;
};

template<> struct Length<Empty> {
  static const int value = 0;
};
\end{lstlisting}

The edge case is written behind the recursive one, that the length value of empty list is 0.
In recursive case, \verb|List::rest| accesses the sub-list of \verb|L|, and \verb|Length<L::rest>::value|
will be recursively calulcated by the C++ compiler. Note that the \verb|typename| keywords should
be inserted before \verb|L::res| to indicate it is a type. After that we add it by 1 as the final
value of the length.

The \verb|Length| calculator can be used as below.

\begin{lstlisting}
cout<<Length<A>::value;
\end{lstlisting}

Where list type \verb|A| is defined in previous section.

How to testing if two lists are identical is left as exercise to the reader.

\subsection{Mutating}

Strictly speaking, we can't mutate the list at all in purely funcitonal settings. What we mutates in
imperative settings are just creation of new lists. Almost all functional environment support gabbage
collection, the orginal list may either persisited or reusing, or released (dropped) at sometime \cite{okasaki-book}(Chapter 2).

\subsubsection{Appending}
Function $cons$ can be viewed as building list by insertion element always on head. If we chaines multiple
$cons$ operation, it repeatedly constructs a list from right to the left. Appending on the other hand,
is an operation adding element to the tail. Compare to $cons$ which is trivial constant time $O(1)$ operation,
We must traverse the whole list to locate the appending position, it means that appending is bound to
$O(N)$, where $N$ is the length of the list. In order to speed up the appending, imperative implementation
typically use a field (variable) to record the tail position of a list, so that the traversing can be
avoided. However, in purely functional settings we can't use such `tail' pointer. The appending has to
be realized in recursive manner.

\be
append(L, x) = \left \{
  \begin{array}
  {r@{\quad:\quad}l}
  \{ x \} & L = \Phi \\
  cons(first(L), append(rest(L), x)) & otherwise
  \end{array}
\right.
\ee

That the algoritm handles two different appending cases.
\begin{itemize}
\item If the list is empty, the result is a singleton list contains $x$, which is the element to be appended. The singleton list denotion $\{ x \} = cons(x, \Phi)$, is a simplified form of $cons$ the element with an empty list $\Phi$;
\item Otherwise, for the none-empty list, the result can be achieved by first appending the element $x$ to the rest sub-list, then construct the first element of $L$ with the recursive appending result. 
\end{itemize}

For the none-trivial case, if we denote $L= \{l_0, l_1, ... \}$, and $L' = \{ l_2, l_3, ...\}$ the equation can be
written as.

\be
append(L, x) = \left \{
  \begin{array}
  {r@{\quad:\quad}l}
  \{ x \} & L = \Phi \\
  cons(l_0, append(L', x)) & otherwise
  \end{array}
\right.
\ee

We'll use both forms in the rest of this appendix.

The following Scheme/Lisp program implements this algorithm.

\lstset{language=Lisp}
\begin{lstlisting}
(define (append lst x)
  (if (null? lst) 
      (list x) 
      (cons (car lst) (append (cdr lst) x))))
\end{lstlisting}

Even without the tail pointer, it's possible to traverse the list imperatively and append the element at the end.

\begin{algorithmic}
\Function{Append}{$L, x$}
  \If{$L = $ NIL}
    \State $L \gets$ \Call{Cons}{$x$, NIL}
  \Else
    \While{\Call{Rest}{$L$} $\neq$ NIL}
      \State $L \gets$ \Call{Rest}{$L$}
    \EndWhile
    \State \Call{Rest}{$L$} $\gets$ \Call{Cons}{$x$, NIL}
  \EndIf
  \State \Return $L$
\EndFunction
\end{algorithmic}

The following ISO C++ programs implements this algorithm. How to utilize a tail field to speed up the appending
is left as exercise to the reader for interesting.

\lstset{language=C++}
\begin{lstlisting}
template<typename T>
List<T>* append(List<T>* xs, T x) {
  List<T> *tail, *head;
  for (head = tail = xs; xs; xs = xs->next)
    tail = xs;
  if (!head)
    head = cons<T>(x, NULL);
  else
    tail->next = cons<T>(x, NULL);
  return head;
}
\end{lstlisting}

For purely interesting purpose, the C++ template meta programming example is given for appending algorithm
as well.

\begin{lstlisting}
template<typename L, int x> struct Append {
  typedef List<L::first, typename Append<typename L::rest, x>::result> result;
};

template<int x> struct Append<Empty, x> {
  typedef List<x, Empty> result;
};
\end{lstlisting}

The edge case is written behind the recursive case as same. That appending an element x to empty list
results a singleton; otherwise, the element is firstly appended to the rest of the list, and we
construct the first element with this recusive appending result to get the final list.

\subsubsection{insertion}

include insertion sort

\subsection{deletion}

\subsection{Access the last element}
Although accessing the first element and the rest list $L'$ is trivial, the opposite operation, that
retreiving the last element and the initial sub list needs linear time. If the list isn't empty, we
need traverse it till the tail to get these two components. Below are their imperative descriptions.

\begin{algorithmic}
\Function{Last}{$L$}
  \State $x \gets $ NIL
  \While{$L \neq$ NIL}
    \State $x \gets $ \Call{First}{$L$}
    \State $L \gets $ \Call{Rest}{$L$}
  \EndWhile
  \State \Return $x$
\EndFunction
\Statex
\Function{Init}{$L$}
  \State $L' \gets $ NIL
  \While{$L \neq$ NIL}
    \State $L' \gets$ \textproc{Append}($L'$, \Call{First}{$L$})
    \State $L \gets $ \Call{Rest}{$L$}
  \EndWhile
  \State \Return $L'$
\EndFunction
\end{algorithmic}

\subsection{concatenate}

\subsection{sum and product}

\subsection{maximum and minimum}

\begin{Exercise}
\begin{itemize}
\item Given two lists $L_1$ and $L_2$, design a algorithm $eq(L_1, L_2)$ to test if they are equal to each other.
Here equality means the lengths are same, and at the same time, every elements in both list are identical.
\item Augment the list with a `tail' field, so that the appending algorithm can be realized in constant $O(1)$ time but
not linear $O(N)$ time. Feel free to choose your favorite imperative programming language. Please don't refer to the
example source code along with this book before you try it.
\item With `tail' field augmented to list, for which list operations this field must be updated? How it affects to the
performance?
\item If you know about C++ template programming, write a program to print a list.
\end{itemize}
\end{Exercise}

\section{Transformation}

\subsection{mapping}

\subsection{reverse}

\section{Extract sub-lists}

\section{take and drop}

\section{split at and breaking and grouping}

\section{Folding}

\subsection{folding from left}

\subsection{folding from right}

\subsection{folding in practice}

\subsubsection{concatenate a list of list}

\section{Searching and matching}

\subsection{Existence testing}

\subsection{Looking up}

\subsection{finding and filtering}

\subsection{indexing}

\subsection{Matching}

prefix, postfix, and infix

\section{zipping and unzipping}

% ================================================================
%                 Short summary
% ================================================================
\section{Notes and short summary}
...

% ================================================================
%                 Appendix
% ================================================================

\begin{thebibliography}{99}

\bibitem{moderncxx}
Andrei Alexandrescu. ``Modern C++ design: Generic Programming and Design Patterns Applied''. Addison Wesley February 01, 2001, ISBN 0-201-70431-5

\bibitem{mittype}
Benjamin C. Pierce. ``Types and Programming Languages''. The MIT Press, 2002. ISBN:0262162091

\bibitem{SICP}
Harold Abelson, Gerald Jay Sussman, Julie Sussman. ``Structure and Interpretation of Computer Programs, 2nd Edition''. MIT Press, 1996, ISBN 0-262-51087-1

\bibitem{okasaki-book}
Chris Okasaki. ``Purely Functional Data Structures''. Cambridge university press, (July 1, 1999), ISBN-13: 978-0521663502

\end{thebibliography}

\ifx\wholebook\relax \else
\end{document}
\fi

% LocalWords:  typedef struct typename
