\ifx\wholebook\relax \else
% ------------------------ 

\documentclass{article}
%------------------- Other types of document example ------------------------
%
%\documentclass[twocolumn]{IEEEtran-new}
%\documentclass[12pt,twoside,draft]{IEEEtran}
%\documentstyle[9pt,twocolumn,technote,twoside]{IEEEtran}
%
%-----------------------------------------------------------------------------
%%
% loading packages
%
\newif\ifpdf
\ifx\pdfoutput\undefined % We're not running pdftex
  \pdffalse
\else
  \pdftrue
\fi
%
%
\ifpdf
  \RequirePackage[pdftex,%
            CJKbookmarks,%
       bookmarksnumbered,%
              colorlinks,%
          linkcolor=blue,%
              hyperindex,%
        plainpages=false,%
       pdfstartview=FitH]{hyperref}
\else
  \RequirePackage[dvipdfm,%
             CJKbookmarks,%
        bookmarksnumbered,%
               colorlinks,%
           linkcolor=blue,%
               hyperindex,%
         plainpages=false,%
        pdfstartview=FitH]{hyperref}
  \AtBeginDvi{\special{pdf:tounicode GBK-EUC-UCS2}} % GBK -> Unicode
\fi
\usepackage{hyperref}

% other packages
%-----------------------------------------------------------------------------
\usepackage{graphicx, color}
\usepackage{CJK}
%
% for programming 
%
\usepackage{verbatim}
\usepackage{listings}


\lstdefinelanguage{Smalltalk}{
  morekeywords={self,super,true,false,nil,thisContext}, % This is overkill
  morestring=[d]',
  morecomment=[s]{"}{"},
  alsoletter={\#:},
  escapechar={!},
  literate=
    {BANG}{!}1
    {UNDERSCORE}{\_}1
    {\\st}{Smalltalk}9 % convenience -- in case \st occurs in code
    % {'}{{\textquotesingle}}1 % replaced by upquote=true in \lstset
    {_}{{$\leftarrow$}}1
    {>>>}{{\sep}}1
    {^}{{$\uparrow$}}1
    {~}{{$\sim$}}1
    {-}{{\sf -\hspace{-0.13em}-}}1  % the goal is to make - the same width as +
    %{+}{\raisebox{0.08ex}{+}}1		% and to raise + off the baseline to match -
    {-->}{{\quad$\longrightarrow$\quad}}3
	, % Don't forget the comma at the end!
  tabsize=2
}[keywords,comments,strings]

\lstloadlanguages{C++, Lisp, Haskell, Python, Smalltalk}

% ======================================================================

\def\BibTeX{{\rm B\kern-.05em{\sc i\kern-.025em b}\kern-.08em
    T\kern-.1667em\lower.7ex\hbox{E}\kern-.125emX}}

\newtheorem{theorem}{Theorem}

%
% mathematics
%
\newcommand{\be}{\begin{equation}}
\newcommand{\ee}{\end{equation}}
\newcommand{\bmat}[1]{\left( \begin{array}{#1} }
\newcommand{\emat}{\end{array} \right) }
\newcommand{\VEC}[1]{\mbox{\boldmath $#1$}}

% numbered equation array
\newcommand{\bea}{\begin{eqnarray}}
\newcommand{\eea}{\end{eqnarray}}

% equation array not numbered
\newcommand{\bean}{\begin{eqnarray*}}
\newcommand{\eean}{\end{eqnarray*}}

\RequirePackage{CJK,CJKnumb,CJKulem,CJKpunct}
% we use CJK as default environment
\AtBeginDocument{\begin{CJK*}{GBK}{song}\CJKtilde\CJKindent\CJKcaption{GB}}
\AtEndDocument{\clearpage\end{CJK*}}

%
% loading packages
%

\RequirePackage{ifpdf}

%
%
\ifpdf
  \RequirePackage[pdftex,%
       bookmarksnumbered,%
              colorlinks,%
          linkcolor=blue,%
              hyperindex,%
        plainpages=false,%
       pdfstartview=FitH]{hyperref}
\else
  \RequirePackage[dvipdfm,%
        bookmarksnumbered,%
               colorlinks,%
           linkcolor=blue,%
               hyperindex,%
         plainpages=false,%
        pdfstartview=FitH]{hyperref}
\fi
\usepackage{hyperref}

% other packages
%--------------------------------------------------------------------------
\usepackage{graphicx, color}
\usepackage{subfig}

\usepackage{amsmath, amsthm, amssymb} % for math
\usepackage{exercise} % for exercise

%
% for programming 
%
\usepackage{verbatim}
\usepackage{listings}
%\usepackage{algorithmic} %old version; we can use algorithmicx instead
\usepackage{algorithm} 
\usepackage[noend]{algpseudocode} %for pseudo code, include algorithmicsx automatically
\usepackage{makeidx} % for index support


\lstdefinelanguage{Smalltalk}{
  morekeywords={self,super,true,false,nil,thisContext}, % This is overkill
  morestring=[d]',
  morecomment=[s]{"}{"},
  alsoletter={\#:},
  escapechar={!},
  literate=
    {BANG}{!}1
    {UNDERSCORE}{\_}1
    {\\st}{Smalltalk}9 % convenience -- in case \st occurs in code
    % {'}{{\textquotesingle}}1 % replaced by upquote=true in \lstset
    {_}{{$\leftarrow$}}1
    {>>>}{{\sep}}1
    {^}{{$\uparrow$}}1
    {~}{{$\sim$}}1
    {-}{{\sf -\hspace{-0.13em}-}}1  % the goal is to make - the same width as +
    %{+}{\raisebox{0.08ex}{+}}1		% and to raise + off the baseline to match -
    {-->}{{\quad$\longrightarrow$\quad}}3
	, % Don't forget the comma at the end!
  tabsize=2
}[keywords,comments,strings]

\lstloadlanguages{C++, Lisp, Haskell, Python, Smalltalk}

% ======================================================================

\def\BibTeX{{\rm B\kern-.05em{\sc i\kern-.025em b}\kern-.08em
    T\kern-.1667em\lower.7ex\hbox{E}\kern-.125emX}}

%
% mathematics
%
\newcommand{\be}{\begin{equation}}
\newcommand{\ee}{\end{equation}}
\newcommand{\bmat}[1]{\left( \begin{array}{#1} }
\newcommand{\emat}{\end{array} \right) }
\newcommand{\VEC}[1]{\mbox{\boldmath $#1$}}

% numbered equation array
\newcommand{\bea}{\begin{eqnarray}}
\newcommand{\eea}{\end{eqnarray}}

% equation array not numbered
\newcommand{\bean}{\begin{eqnarray*}}
\newcommand{\eean}{\end{eqnarray*}}

\newtheorem{theorem}{Theorem}[section]
\newtheorem{lemma}[theorem]{Lemma}
\newtheorem{proposition}[theorem]{Proposition}
\newtheorem{corollary}[theorem]{Corollary}


\setcounter{page}{1}

\begin{document}

\fi
%--------------------------

% ================================================================
%                 COVER PAGE
% ================================================================

\title{AVL tree}

\author{Liu~Xinyu
\thanks{{\bfseries Liu Xinyu } \newline
  Email: liuxinyu95@gmail.com \newline}
  }

\markboth{AVL tree}{AlgoXY}

\maketitle

\ifx\wholebook\relax
\chapter{AVL tree}
\fi

% ================================================================
%                 Introduction
% ================================================================
\section{Introduction}
\label{introduction} \index{AVL tree}

\subsection{How to measure the balance of a tree?}
Besides red-black tree, are there any other intuitive solutions of self-balancing
binary search tree? In order to measure how balancing a binary search tree,
one idea is to compare the height of the left sub-tree and right sub-tree.
If they differs a lot, the tree isn't well balanced. Let's denote the 
difference height between two children as below

\be
  \delta(T) = |L| - |R|
\ee

Where $|T|$ means the height of tree $T$, and $L$, $R$ denotes the left
sub-tree and right sub-tree.

If $\delta(T) = 0$, The tree is defnitely balanced. For example, a 
complete binary tree has $N=2^h-1$ nodes for height $h$. There is
no empty branches unless the leafs. Another trivial case is empty 
tree. $\delta(\phi) = 0$. The less absolute value of $\delta(T)$
the more balancing the tree is. 

% ================================================================
% Definition
% ================================================================
\section{Definition of AVL tree}
\index{AVL tree!definition}

TODO: math definition and data layout, 
C++ and Haskell definitions.

% ================================================================
%                 Insertion
% ================================================================
\section{Insertion}
\index{AVL tree!insertion}

TODO:
  Pattern matching insertion
  Traditional insertion
 
% ================================================================
%                 Deletion
% ================================================================

\section{Deletion}
\index{AVL tree!deletion}

  TODO: for functional setting, no need??? only imperative one.


\section{Short summary}
  TODO: short summary and introduction to the following chatpers.

\begin{thebibliography}{99}

\bibitem{CLRS}
Thomas H. Cormen, Charles E. Leiserson, Ronald L. Rivest and Clifford Stein. 
``Introduction to Algorithms, Second Edition''. ISBN:0262032937. The MIT Press. 2001

\bibitem{okasaki}
Chris Okasaki. ``FUNCTIONAL PEARLS Red-Black Trees in a Functional Setting''. J. Functional Programming. 1998

\bibitem{wiki}
Wikipedia. ``AVL tree''. http://en.wikipedia.org/wiki/AVL\_tree

\end{thebibliography}

\ifx\wholebook\relax\else
\end{document}
\fi
