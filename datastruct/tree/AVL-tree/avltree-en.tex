\ifx\wholebook\relax \else
% ------------------------ 

\documentclass{article}
%------------------- Other types of document example ------------------------
%
%\documentclass[twocolumn]{IEEEtran-new}
%\documentclass[12pt,twoside,draft]{IEEEtran}
%\documentstyle[9pt,twocolumn,technote,twoside]{IEEEtran}
%
%-----------------------------------------------------------------------------
%%
% loading packages
%
\newif\ifpdf
\ifx\pdfoutput\undefined % We're not running pdftex
  \pdffalse
\else
  \pdftrue
\fi
%
%
\ifpdf
  \RequirePackage[pdftex,%
            CJKbookmarks,%
       bookmarksnumbered,%
              colorlinks,%
          linkcolor=blue,%
              hyperindex,%
        plainpages=false,%
       pdfstartview=FitH]{hyperref}
\else
  \RequirePackage[dvipdfm,%
             CJKbookmarks,%
        bookmarksnumbered,%
               colorlinks,%
           linkcolor=blue,%
               hyperindex,%
         plainpages=false,%
        pdfstartview=FitH]{hyperref}
  \AtBeginDvi{\special{pdf:tounicode GBK-EUC-UCS2}} % GBK -> Unicode
\fi
\usepackage{hyperref}

% other packages
%-----------------------------------------------------------------------------
\usepackage{graphicx, color}
\usepackage{CJK}
%
% for programming 
%
\usepackage{verbatim}
\usepackage{listings}


\lstdefinelanguage{Smalltalk}{
  morekeywords={self,super,true,false,nil,thisContext}, % This is overkill
  morestring=[d]',
  morecomment=[s]{"}{"},
  alsoletter={\#:},
  escapechar={!},
  literate=
    {BANG}{!}1
    {UNDERSCORE}{\_}1
    {\\st}{Smalltalk}9 % convenience -- in case \st occurs in code
    % {'}{{\textquotesingle}}1 % replaced by upquote=true in \lstset
    {_}{{$\leftarrow$}}1
    {>>>}{{\sep}}1
    {^}{{$\uparrow$}}1
    {~}{{$\sim$}}1
    {-}{{\sf -\hspace{-0.13em}-}}1  % the goal is to make - the same width as +
    %{+}{\raisebox{0.08ex}{+}}1		% and to raise + off the baseline to match -
    {-->}{{\quad$\longrightarrow$\quad}}3
	, % Don't forget the comma at the end!
  tabsize=2
}[keywords,comments,strings]

\lstloadlanguages{C++, Lisp, Haskell, Python, Smalltalk}

% ======================================================================

\def\BibTeX{{\rm B\kern-.05em{\sc i\kern-.025em b}\kern-.08em
    T\kern-.1667em\lower.7ex\hbox{E}\kern-.125emX}}

\newtheorem{theorem}{Theorem}

%
% mathematics
%
\newcommand{\be}{\begin{equation}}
\newcommand{\ee}{\end{equation}}
\newcommand{\bmat}[1]{\left( \begin{array}{#1} }
\newcommand{\emat}{\end{array} \right) }
\newcommand{\VEC}[1]{\mbox{\boldmath $#1$}}

% numbered equation array
\newcommand{\bea}{\begin{eqnarray}}
\newcommand{\eea}{\end{eqnarray}}

% equation array not numbered
\newcommand{\bean}{\begin{eqnarray*}}
\newcommand{\eean}{\end{eqnarray*}}

\RequirePackage{CJK,CJKnumb,CJKulem,CJKpunct}
% we use CJK as default environment
\AtBeginDocument{\begin{CJK*}{GBK}{song}\CJKtilde\CJKindent\CJKcaption{GB}}
\AtEndDocument{\clearpage\end{CJK*}}

%
% loading packages
%

\RequirePackage{ifpdf}

%
%
\ifpdf
  \RequirePackage[pdftex,%
       bookmarksnumbered,%
              colorlinks,%
          linkcolor=blue,%
              hyperindex,%
        plainpages=false,%
       pdfstartview=FitH]{hyperref}
\else
  \RequirePackage[dvipdfm,%
        bookmarksnumbered,%
               colorlinks,%
           linkcolor=blue,%
               hyperindex,%
         plainpages=false,%
        pdfstartview=FitH]{hyperref}
\fi
\usepackage{hyperref}

% other packages
%--------------------------------------------------------------------------
\usepackage{graphicx, color}
\usepackage{subfig}

\usepackage{amsmath, amsthm, amssymb} % for math
\usepackage{exercise} % for exercise

%
% for programming 
%
\usepackage{verbatim}
\usepackage{listings}
%\usepackage{algorithmic} %old version; we can use algorithmicx instead
\usepackage{algorithm} 
\usepackage[noend]{algpseudocode} %for pseudo code, include algorithmicsx automatically
\usepackage{makeidx} % for index support


\lstdefinelanguage{Smalltalk}{
  morekeywords={self,super,true,false,nil,thisContext}, % This is overkill
  morestring=[d]',
  morecomment=[s]{"}{"},
  alsoletter={\#:},
  escapechar={!},
  literate=
    {BANG}{!}1
    {UNDERSCORE}{\_}1
    {\\st}{Smalltalk}9 % convenience -- in case \st occurs in code
    % {'}{{\textquotesingle}}1 % replaced by upquote=true in \lstset
    {_}{{$\leftarrow$}}1
    {>>>}{{\sep}}1
    {^}{{$\uparrow$}}1
    {~}{{$\sim$}}1
    {-}{{\sf -\hspace{-0.13em}-}}1  % the goal is to make - the same width as +
    %{+}{\raisebox{0.08ex}{+}}1		% and to raise + off the baseline to match -
    {-->}{{\quad$\longrightarrow$\quad}}3
	, % Don't forget the comma at the end!
  tabsize=2
}[keywords,comments,strings]

\lstloadlanguages{C++, Lisp, Haskell, Python, Smalltalk}

% ======================================================================

\def\BibTeX{{\rm B\kern-.05em{\sc i\kern-.025em b}\kern-.08em
    T\kern-.1667em\lower.7ex\hbox{E}\kern-.125emX}}

%
% mathematics
%
\newcommand{\be}{\begin{equation}}
\newcommand{\ee}{\end{equation}}
\newcommand{\bmat}[1]{\left( \begin{array}{#1} }
\newcommand{\emat}{\end{array} \right) }
\newcommand{\VEC}[1]{\mbox{\boldmath $#1$}}

% numbered equation array
\newcommand{\bea}{\begin{eqnarray}}
\newcommand{\eea}{\end{eqnarray}}

% equation array not numbered
\newcommand{\bean}{\begin{eqnarray*}}
\newcommand{\eean}{\end{eqnarray*}}

\newtheorem{theorem}{Theorem}[section]
\newtheorem{lemma}[theorem]{Lemma}
\newtheorem{proposition}[theorem]{Proposition}
\newtheorem{corollary}[theorem]{Corollary}


\setcounter{page}{1}

\begin{document}

\fi
%--------------------------

% ================================================================
%                 COVER PAGE
% ================================================================

\title{AVL tree}

\author{Liu~Xinyu
\thanks{{\bfseries Liu Xinyu } \newline
  Email: liuxinyu95@gmail.com \newline}
  }

\markboth{AVL tree}{AlgoXY}

\maketitle

\ifx\wholebook\relax
\chapter{AVL tree}
\fi

% ================================================================
%                 Introduction
% ================================================================
\section{Introduction}
\label{introduction} \index{AVL tree}

\subsection{How to measure the balance of a tree?}
Besides red-black tree, are there any other intuitive solutions of self-balancing
binary search tree? In order to measure how balancing a binary search tree,
one idea is to compare the height of the left sub-tree and right sub-tree.
If they differs a lot, the tree isn't well balanced. Let's denote the 
difference height between two children as below

\be
  \delta(T) = |L| - |R|
\ee

Where $|T|$ means the height of tree $T$, and $L$, $R$ denotes the left
sub-tree and right sub-tree.

If $\delta(T) = 0$, The tree is defnitely balanced. For example, a 
complete binary tree has $N=2^h-1$ nodes for height $h$. There is
no empty branches unless the leafs. Another trivial case is empty 
tree. $\delta(\phi) = 0$. The less absolute value of $\delta(T)$
the more balancing the tree is. 

We define $\delta(T)$ as the {\em balance factor} of a binary search
tree.

% ================================================================
% Definition
% ================================================================
\section{Definition of AVL tree}
\index{AVL tree!definition}

An AVL tree is a special binary search tree, that all sub-trees 
satisfying the following criteria.

\be
  |\delta(T)| \leq 1
\ee

The absolute value of balance factor is less than or equal to 1, which
means there are only three valid values, -1, 0 and 1.

Why AVL tree can keep the tree balanced? In other words, Can this definition
ensure the height of the tree as $O(\lg N)$ where $N$ is the number of
the nodes in the tree? Let's prove this fact.

For an AVL tree of height $h$, The number of nodes varies. It can have at
most $2^h-1$ nodes for a complete binary tree. We are interesting about
how many nodes there are at least. Let's denote the minimum number of nodes
for height $h$ AVL tree as $N(h)$. It's obvious for the trivial cases 
as below.

\begin{itemize}
\item For empty tree, $h=0$, $N(0)=0$;
\item For a singlton root, $h=1$, $N(1)=1$;
\end{itemize}

What's the situation for common case $N(h)$? Figure \ref{fig:N-h-relation}
shows an AVL tree $T$ of height $h$. It contains three part, the root node,
and two sub trees $A, B$. We have the following fact.

\be
  h= max(height(L), height(R)) + 1
\ee

We immediately know that, there must be one child has height $h-1$. Let's 
say $height(A) = h-1$. According to the definition of AVL tree, we have.
$|height(A)-height(B)| \leq 1$. This leads to the fact that the height of
other tree $B$ can't be lower than $h-2$, So the total number of the nodes
of $T$ is the number of nodes in tree $A$, and $B$ plus 1 (for the root node).
We exclaim that.

\be
  N(h) = N(h-1) + N(h-2) + 1
  \label{eq:Fibonacci-like}
\ee

\begin{figure}[htbp]
   \centering
   \includegraphics[scale=0.5]{img/Nh-lvr.ps}
   \caption{An AVL tree with height $h$, one of the sub-tree with height $h-1$, the other is $h-2$} \label{fig:N-h-relation}
\end{figure}

This recusion reminds us the famous Fibonacci series. Actually we can 
transform it to Fibonacci series by defining $N'(h) = N(h)+1$. So equation
\ref{eq:Fibonacci-like} changes to.

\be
  N'(h) = N'(h-1) + N'(h-2)
\ee

\begin{lemma}
\label{lemma:N-phi}
Let $N(h)$ be the minimum number of nodes for an AVL tree with
height $h$. and $N'(h) = N(h) + 1$, then
\be
  N'(h) \geq \phi^h
\ee

Where $\phi = \frac{\sqrt{5}+1}{2}$ is the golden ratio.
\end{lemma}

\begin{proof}
For the trivial case, we have
\begin{itemize}
\item $h=0$, $N'(0) = 1 \geq \phi^0 = 1$
\item $h=1$, $N'(1) = 2 \geq \phi^1 = 1.618...$
\end{itemize}

For the induction case, suppose $N'(h) \geq \phi^h$.
\[
  \begin{array}{lll}
  N'(h+1) & = N'(h) + N'(h-1) & \{Fibonacci\} \\
          & \geq \phi^h + \phi^{h-1} & \\
          & = \phi^{h-1}(\phi + 1) & \{\phi + 1 = \phi^2 = \frac{\sqrt{5}+3}{2}\} \\
          & = \phi^{h+1}
 \end{array}
\]
\end{proof}

From Lemma \ref{lemma:N-phi}, we immedately get

\be
  h \leq log_{\phi}(N+1) = log_{\phi}(2) \cdot \lg (N+1) \approx 1.44 \lg (N+1)
\ee

It tells that the height of AVL tree is proportion to $O(\lg N)$, which
means that AVL tree is balanced.

During the basic mutable tree operations such as insertion and deletion,
if the balance factor changes to any invalide value, some fixing has
to be performed to resume $|\delta|$ within 1. Most implementations utilize
tree rotations. In this chapter, we'll show the pattern matching solution
which is inspired by Okasaki's red-black tree solution\cite{okasaki}.
Because of this modify-fixing approach, AVL tree is also a kind of 
self-balancing binary search tree. For comparison purpose, we'll also
show the procedural algorithms.

Of course we can compute the $\delta$ value recursively, another option
is to store the balance factor inside each nodes, and update them
when we modify the tree. The latter one avoid computing the same value
every time.

Based on this idea, we can add one data field $\delta$ to the original
binary search tree as the following C++ code exmample.

\lstset{language=C++}
\begin{lstlisting}
template <class T>
struct node{
  int delta;
  T key;
  node* left;
  node* right;
  node* parent;
};
\end{lstlisting}

In purely functional setting, some implementation use different 
constructor to store the $\delta$ information. for example in 
\cite{hackage}, there are 4 constructors, E, N, P, Z defined.
E for empty tree, N for tree with negative 1 balance factor,
P for tree with positive 1 balance factor and Z for zero case.

In this chapter, we'll explicitly store the balance factor inside
the node.

\lstset{language=Haskell}
\begin{lstlisting}
data AVLTree a = Empty
               | Br (AVLTree a) a (AVLTree a) Int 
\end{lstlisting}

The immutable operations, including looking up, finding the maximum
and minimum elements are all same as the binary search tree. We'll
skip them and focus on the mutable operations.

% ================================================================
%                 Insertion
% ================================================================
\section{Insertion}
\index{AVL tree!insertion}

Insert a new element to an AVL tree may vialate the AVL tree property
that the $\delta$ absolute value exceeds 1. To resume it, one option
is to do the tree rotation according to the different insertion cases.
Most implementation is based on this approach

Another way is to use the similar pattern matching method mentioned by
Okasaki in his red-black tree implementation \cite{okasaki}. Inspried 
by this idea, it is possible to provide a simiple and intuitive
solution.  

TODO:
  Pattern matching insertion
  Traditional insertion
 
% ================================================================
%                 Deletion
% ================================================================

\section{Deletion}
\index{AVL tree!deletion}

  TODO: for functional setting, no need??? only imperative one.


\section{Chapter note}
AVL tree was invented in 1962 by Adelson-Velskii and Landis\cite{wiki}, 
\cite{TFATP}. The name AVL tree comes from the two inventors's name.

TODO: compare with Red-black tree.

  TODO: short summary and introduction to the following chatpers.

\begin{thebibliography}{99}

\bibitem{hackage}
Data.Tree.AVL http://hackage.haskell.org/packages/archive/AvlTree/4.2/doc/html/Data-Tree-AVL.html

\bibitem{okasaki}
Chris Okasaki. ``FUNCTIONAL PEARLS Red-Black Trees in a Functional Setting''. J. Functional Programming. 1998

\bibitem{wiki}
Wikipedia. ``AVL tree''. http://en.wikipedia.org/wiki/AVL\_tree

\bibitem{TFATP}
Guy Cousinear, Michel Mauny. ``The Functional Approach to Programming''. Cambridge University Press; English Ed edition (October 29, 1998). ISBN-13: 978-0521576819

\end{thebibliography}

\ifx\wholebook\relax\else
\end{document}
\fi
